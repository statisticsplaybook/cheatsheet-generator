\documentclass[10pt,quotespacing]{oblivoir}

% Package imports.
\usepackage{fapapersize}
\usepackage{kotex}
\usepackage{multicol}
\usepackage{calc}
\usepackage{amsmath,amsthm,amsfonts,amssymb}
\usepackage{color,graphicx,overpic}
\usepackage{hyperref}
\usepackage[utf8]{inputenc}
\usepackage{textcomp} % provide euro and other symbols
\usepackage{setspace}\singlespacing
\usepackage[most]{tcolorbox}

$if(highlighting-macros)$
$highlighting-macros$
$endif$

$if(mainfont)$
  \setmainfont[$for(mainfontoptions)$$mainfontoptions$$sep$,$endfor$]{$mainfont$}
$endif$


% 편집용지(가로), 편집용지(세로), 왼,오른,위,아래
\usefapapersize{257mm, 188mm, 10mm, 10mm, 10mm, 10mm}

\hypersetup{
    colorlinks=true,
    linkcolor=blue,
    filecolor=magenta,      
    urlcolor=cyan,
}

\pagestyle{empty}

% No section numbers.
\setcounter{secnumdepth}{0}

% Minimal paragraph indenting and spacing.
\setlength{\parindent}{0pt}
\setlength{\parskip}{0pt plus 0.5ex}
\setlength{\OuterFrameSep}{-0.1pt}

\makeatletter
\preto{\@verbatim}{\topsep=0pt \partopsep=0pt }
\makeatother

\begin{document}

\textbf{$packagename$} Cheatsheet by \href{url}{Issac Lee}

\begin{multicols*}{4}

% Can play around with these as desired.
% \setlength{\columnseprule}{0.25pt}
\setlength{\premulticols}{0.25pt}
\setlength{\postmulticols}{0.25pt}
\setlength{\multicolsep}{0.25pt}
\setlength{\columnsep}{0.25pt}

% This is the "magic" pandoc variable. (This is where your Rmarkdown document is inserted.)
\tiny

\begin{tcolorbox}

$body$

\end{tcolorbox}

% `\end` statements to match the `\begin`s.
\end{multicols*}

\end{document}

